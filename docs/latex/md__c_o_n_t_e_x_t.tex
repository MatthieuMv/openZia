Each module of the pipeline implement various callbacks that parse and transform an {\bfseries{\mbox{\hyperlink{classo_z_1_1_h_t_t_p_1_1_request}{o\+Z\+::\+H\+T\+T\+P\+::\+Request}}}} to create an {\bfseries{\mbox{\hyperlink{classo_z_1_1_h_t_t_p_1_1_response}{o\+Z\+::\+H\+T\+T\+P\+::\+Response}}}}. Thus, the pipeline need a way to store each information though the various modules\+: {\bfseries{\mbox{\hyperlink{classo_z_1_1_context}{o\+Z\+::\+Context}}}}.

\subsection*{Context creation}

Thanks to its \mbox{\hyperlink{classo_z_1_1_packet}{o\+Z\+::\+Packet}} and \mbox{\hyperlink{classo_z_1_1_context_a27f05cdc375f513979a6080896cf9496}{o\+Z\+::\+Context\+::\+Metadata\+Map}} members, a context can store a lot of data about a client. Thus, contexts should be stored in a client structure, in order to reuse it in various callbacks. It allows your server to be way more efficient in memory managment !


\begin{DoxyCode}{0}
\DoxyCodeLine{ \{C++\}}
\DoxyCodeLine{// Client structure}
\DoxyCodeLine{struct Client}
\DoxyCodeLine{\{}
\DoxyCodeLine{    oZ::Context context \{\}; // Client's reusable context}
\DoxyCodeLine{\};}
\end{DoxyCode}


\subsection*{Cache handling}

In this section we will see how caching can be achieved if you really wish to go for performances. The {\bfseries{\mbox{\hyperlink{classo_z_1_1_context}{o\+Z\+::\+Context}}}} class has a {\bfseries{\mbox{\hyperlink{classo_z_1_1_context_a748147258019436983fdbbf6ed51c0b6}{o\+Z\+::\+Context\+::is\+Constant}}}} function that tells if the current context is constant, and thus can be cached. If some of your modules aren\textquotesingle{}t supporting cache for specific H\+T\+TP methods, you can use the {\bfseries{\mbox{\hyperlink{classo_z_1_1_context_ada521ec57fbc2febfd61177e8bbc0128}{o\+Z\+::\+Context\+::not\+Constant}}}} function to set the constant state to false. It\textquotesingle{}s up to you to implement caching. You can accomplish in by creating a module or by handling it directly in the server. 