A Module is a dynamic library which contains a single class loaded at runtime. It means that you must compile every module independently. For that you must implement your modules deriving from {\bfseries{\mbox{\hyperlink{classo_z_1_1_i_module}{o\+Z\+::\+I\+Module}}}} interface. Last, you will need an instantiation function\+: 
\begin{DoxyCode}{0}
\DoxyCodeLine{ \{C++\}}
\DoxyCodeLine{extern "C" IModule *CreateModule(void) \{ return MyModule(); \}}
\DoxyCodeLine{}
\DoxyCodeLine{// you may want to use the OPEN\_ZIA\_MAKE\_ENTRY\_POINT macro:}
\DoxyCodeLine{OPEN\_ZIA\_MAKE\_ENTRY\_POINT(MyModule);}
\end{DoxyCode}
 The A\+PI already does the dynamic library loading for you so you can focus more on creating modules.

\subsection*{Module}

In this section we will take a closer look at the interface class {\bfseries{\mbox{\hyperlink{classo_z_1_1_i_module}{o\+Z\+::\+I\+Module}}}}, and all its components. 
\begin{DoxyCode}{0}
\DoxyCodeLine{ \{C++\}}
\DoxyCodeLine{// Module interface}
\DoxyCodeLine{class IModule;}
\DoxyCodeLine{// Module polymorphic instance}
\DoxyCodeLine{using ModulePtr = std::shared\_ptr<IModule>}
\end{DoxyCode}
 We use shared pointers to store instantiated modules. This allows holding a reference to a module while asserting its existence.

\subsection*{Modules callbacks}

Each module can register multiple callback functions at any state of the pipeline. The enumeration of each callback is {\bfseries{\mbox{\hyperlink{namespaceo_z_a356b278f7c65def0cae75fca8cae268e}{o\+Z\+::\+State}}}}.

Use the different states and the {\bfseries{\mbox{\hyperlink{namespaceo_z_af05a92eb185d18369e9b4acdcd9dcd12}{o\+Z\+::\+Priority}}}} enumeration to sort modules callback. Each state of the pipeline are triggered in this order\+: \begin{quote}
Before\+Parse -\/$>$ Parse -\/$>$ After\+Parse -\/$>$ Before\+Interpret -\/$>$ Interpret -\/$>$ After\+Interpret -\/$>$ Completed \end{quote}


Each callback must take a {\bfseries{\mbox{\hyperlink{classo_z_1_1_context}{o\+Z\+::\+Context}}}} in parameter and return a boolean. If you {\bfseries{return false}} from the callback, the pipeline will {\bfseries{not trigger}} other modules\textquotesingle{} callbacks from the current {\bfseries{\mbox{\hyperlink{namespaceo_z_a356b278f7c65def0cae75fca8cae268e}{o\+Z\+::\+State}}}} and go straight for the next one. At any time, you if an error is set using {\bfseries{\mbox{\hyperlink{classo_z_1_1_context_a036d993634650ee8414c2f12d49d0204}{o\+Z\+::\+Context\+::set\+Error\+State}}}}, the pipeline will trigger the special {\bfseries{o\+Z\+::\+State\+::\+Error}} sate callback.


\begin{DoxyCode}{0}
\DoxyCodeLine{ \{C++\}}
\DoxyCodeLine{bool MyModule::myCallback(oZ::Context \&context)}
\DoxyCodeLine{\{}
\DoxyCodeLine{    if (hasError(context)) \{}
\DoxyCodeLine{        context.setErrorState();}
\DoxyCodeLine{        return false; // If you return true here, the pipeline will finish to trigger every callback of the current state and not immediatly call Error callbacks}
\DoxyCodeLine{    \}}
\DoxyCodeLine{    return true;}
\DoxyCodeLine{\}}
\end{DoxyCode}


\subsection*{Must have functions}

Each module have a set of virtual function to be overriden. Actually, only 2 of them are pure virtual \+: 
\begin{DoxyCode}{0}
\DoxyCodeLine{ \{C++\}}
\DoxyCodeLine{// Get the raw string name of module instance}
\DoxyCodeLine{virtual const char *getName(void) const = 0;}
\DoxyCodeLine{}
\DoxyCodeLine{// Register module's callbacks to the pipeline}
\DoxyCodeLine{virtual void onRegisterCallbacks(Pipeline \&pipeline) = 0;}
\end{DoxyCode}


\subsubsection*{How\+To\+: H\+T\+TP module}

Let\textquotesingle{}s see a simple example with an independent H\+T\+TP Module\+: 
\begin{DoxyCode}{0}
\DoxyCodeLine{ \{C++\}}
\DoxyCodeLine{class HTTPModule : public oZ::IModule}
\DoxyCodeLine{\{}
\DoxyCodeLine{public:}
\DoxyCodeLine{    // Get the raw string name of module instance}
\DoxyCodeLine{    virtual const char *getName(void) const \{ return "HTTPModule"; \}}
\DoxyCodeLine{}
\DoxyCodeLine{    // Register module's callbacks to the pipeline}
\DoxyCodeLine{    virtual void onRegisterCallbacks(oZ::Pipeline \&pipeline) \{}
\DoxyCodeLine{        pipeline.registerCallback(}
\DoxyCodeLine{            oZ::State::Parse, // Callback triggered when parsing the HTTP request}
\DoxyCodeLine{            oZ::Priority::ASAP, // As soon as possible}
\DoxyCodeLine{            this, \&HTTPModule::onParseHeader // Actual function callback}
\DoxyCodeLine{        );}
\DoxyCodeLine{    \}}
\DoxyCodeLine{    bool onParseHeader(oZ::Context \&context) \{}
\DoxyCodeLine{        const auto \&byteArray = context.getByteArray();}
\DoxyCodeLine{        auto \&request = context.getRequest();}
\DoxyCodeLine{        auto \&header = request.getHeader();}
\DoxyCodeLine{        // Parse 'byteArray' to fill 'header' and 'request' data}
\DoxyCodeLine{        return true; // Tell the pipeline we continue to process this state}
\DoxyCodeLine{    \}}
\DoxyCodeLine{\};}
\end{DoxyCode}
 And that\textquotesingle{}s it ! You don\textquotesingle{}t have to implement anything else to create an independent module.

\subsection*{Optional functions}

However, there are more virtual functions for more complex needs. These functions are default-\/implemented to let you the choice of using them or not. 
\begin{DoxyCode}{0}
\DoxyCodeLine{ \{C++\}}
\DoxyCodeLine{// Callback triggered when a client is connected}
\DoxyCodeLine{virtual void onConnected(Context \&context);}
\DoxyCodeLine{}
\DoxyCodeLine{// Callback triggered when a client is disconnected}
\DoxyCodeLine{virtual void onDisconnected(Context \&context);}
\DoxyCodeLine{}
\DoxyCodeLine{// Callback triggered when a client sent a message}
\DoxyCodeLine{virtual MessageState onMessageAvailable(Context \&context);}
\DoxyCodeLine{}
\DoxyCodeLine{// Get the list of dependencies (as a vector of raw string, see function getName above)}
\DoxyCodeLine{virtual Dependencies getDependencies(void) const noexcept;}
\DoxyCodeLine{}
\DoxyCodeLine{// Given a pipeline reference, find each dependent module to store them internally}
\DoxyCodeLine{virtual void onRetrieveDependencies(Pipeline \&pipeline);}
\DoxyCodeLine{}
\DoxyCodeLine{// Given a directory path (where all configs are), load module's configuration file}
\DoxyCodeLine{virtual void onLoadConfigurationFile(const std::string \&directory);}
\end{DoxyCode}


\subsection*{How\+To\+: Networking}

If you implement networking as a standalone module, you need to use {\bfseries{Pipeline\+::on\+Connection}}, {\bfseries{Pipeline\+::on\+Disconnection}}, {\bfseries{I\+Module\+::on\+Connection}}, {\bfseries{I\+Module\+::on\+Disconnection}}, {\bfseries{I\+Module\+::on\+Message\+Avaible}} to implement it as a standalone module.


\begin{DoxyCode}{0}
\DoxyCodeLine{ \{C++\}}
\DoxyCodeLine{// Networking module}
\DoxyCodeLine{class NetModule : public oZ::IModule}
\DoxyCodeLine{\{}
\DoxyCodeLine{public:}
\DoxyCodeLine{    virtual void onConnection(oZ::Context \&context) \{}
\DoxyCodeLine{        // Do your connection stuff}
\DoxyCodeLine{    \}}
\DoxyCodeLine{}
\DoxyCodeLine{    virtual void onDisconnection(oZ::Context \&context) \{}
\DoxyCodeLine{        // Do your disconnection stuff}
\DoxyCodeLine{    \}}
\DoxyCodeLine{}
\DoxyCodeLine{    virtual MessageState onMessageAvailable(oZ::Context \&context) \{}
\DoxyCodeLine{        // Fill 'context.getPacket().getByteArray()' by reading is socket}
\DoxyCodeLine{        return MessageState::Done; // Returns Ready to tell pipeline that you filled the context}
\DoxyCodeLine{    \}}
\DoxyCodeLine{    // ...}
\DoxyCodeLine{\};}
\DoxyCodeLine{}
\DoxyCodeLine{// Client structure}
\DoxyCodeLine{struct Client}
\DoxyCodeLine{\{}
\DoxyCodeLine{    oZ::Context context \{\}; // Client's reusable context}
\DoxyCodeLine{\};}
\DoxyCodeLine{}
\DoxyCodeLine{void MyServer::onClientConnected(Client \&client)}
\DoxyCodeLine{\{}
\DoxyCodeLine{    \_pipeline.onConnection(client.context);}
\DoxyCodeLine{\}}
\DoxyCodeLine{}
\DoxyCodeLine{void MyServer::onClientDisconnected(Client \&client)}
\DoxyCodeLine{\{}
\DoxyCodeLine{    \_pipeline.onDisconnection(client.context);}
\DoxyCodeLine{\}}
\DoxyCodeLine{}
\DoxyCodeLine{void MyServer::onClientReadable(Client \&client)}
\DoxyCodeLine{\{}
\DoxyCodeLine{    client.context.clear();}
\DoxyCodeLine{    switch (\_pipeline.onMessageAvailable(client.context)) \{}
\DoxyCodeLine{    case MessageState::Readable:}
\DoxyCodeLine{        // The message has not been handled by any module}
\DoxyCodeLine{        break;}
\DoxyCodeLine{    case MessageState::Done:}
\DoxyCodeLine{        // The message has been processed by the pipeline}
\DoxyCodeLine{        break;}
\DoxyCodeLine{    case MessageState::Disconnection:}
\DoxyCodeLine{        // The message was a TCP disconnection}
\DoxyCodeLine{        break;}
\DoxyCodeLine{    \}}
\DoxyCodeLine{\}}
\end{DoxyCode}


\subsubsection*{How\+To\+: Dependencies}

Let\textquotesingle{}s first see how dependencies are handled given a short example\+:


\begin{DoxyCode}{0}
\DoxyCodeLine{ \{C++\}}
\DoxyCodeLine{class ChildModule : public oZ::IModule}
\DoxyCodeLine{\{}
\DoxyCodeLine{public:}
\DoxyCodeLine{    virtual const char *getName(void) const \{ return "Child"; \}}
\DoxyCodeLine{    virtual void onRegisterCallbacks(oZ::Pipeline \&pipeline) \{ ... \}}
\DoxyCodeLine{\};}
\DoxyCodeLine{}
\DoxyCodeLine{class RootModule : public oZ::IModule}
\DoxyCodeLine{\{}
\DoxyCodeLine{public:}
\DoxyCodeLine{    virtual const char *getName(void) const \{ return "Root"; \}}
\DoxyCodeLine{    virtual void onRegisterCallbacks(oZ::Pipeline \&pipeline) \{ ... \}}
\DoxyCodeLine{}
\DoxyCodeLine{    // Return the list of dependencies}
\DoxyCodeLine{    virtual Dependencies getDependencies(void) const noexcept \{ return \{ "Child" \}; \}}
\DoxyCodeLine{}
\DoxyCodeLine{    // Find dependencies and store them before any process for performance reasons}
\DoxyCodeLine{    virtual void onRetrieveDependencies(oZ::Pipeline \&pipeline) \{}
\DoxyCodeLine{        \_child = pipeline.findModule<ChildModule>();}
\DoxyCodeLine{    \}}
\DoxyCodeLine{}
\DoxyCodeLine{private:}
\DoxyCodeLine{    std::shared\_ptr<ChildModule> \_child;}
\DoxyCodeLine{\};}
\end{DoxyCode}


\subsubsection*{How\+To\+: Configuration file}

Let\textquotesingle{}s see how simple it is to add a configuration file to setup a module. Please note that the configuration file language is totally up to you, in this example it is named {\itshape My\+Custom\+Config\+Loader}. 
\begin{DoxyCode}{0}
\DoxyCodeLine{ \{C++\}}
\DoxyCodeLine{// Import your own config loader}
\DoxyCodeLine{\#include "MyCustomConfigLoader.hpp"}
\DoxyCodeLine{}
\DoxyCodeLine{class MyModule : public oZ::IModule}
\DoxyCodeLine{\{}
\DoxyCodeLine{public:}
\DoxyCodeLine{    static const auto *FileName = "MyModule.conf";}
\DoxyCodeLine{}
\DoxyCodeLine{    virtual const char *getName(void) const \{ return "My" \};}
\DoxyCodeLine{    virtual void onRegisterCallbacks(oZ::Pipeline \&pipeline) \{ ... \}}
\DoxyCodeLine{}
\DoxyCodeLine{    // Open configuration file and load some properties using MyCustomConfigLoader}
\DoxyCodeLine{    virtual void onLoadConfigurationFile(const std::string \&directory) \{}
\DoxyCodeLine{        auto config = MyCustomConfigLoader::Parse(directory + FileName);}
\DoxyCodeLine{        \_x = config["x"].toInt();}
\DoxyCodeLine{        \_y = config["y"].toInt();}
\DoxyCodeLine{    \}}
\DoxyCodeLine{}
\DoxyCodeLine{private:}
\DoxyCodeLine{    int \_x = 0, \_y = 0;}
\DoxyCodeLine{\};}
\end{DoxyCode}
